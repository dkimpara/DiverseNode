

\begin{figure}[hbt!]\centering
    \subfloat[$\sd$]{\label{}
    \includegraphics[width=.45\linewidth]{images/change_all-stdd.png}}\par

	\subfloat[$\ss$]{\label{}
		\includegraphics[width=.4\linewidth]{images/ca-rs.png}}\hfill
	\subfloat[$\sw$]{\label{}
		\includegraphics[width=.4\linewidth]{images/ca-rw.png}}

    \caption{3D scatter plot showing the effect on SPL and CD of each
    standard deviation of (a) $d$, (b) $r_s$ and (c) $r_w$. Each dot is the result of one simulation run, colored
    according to the standard deviation value.}
    \label{fig:cAll}
\end{figure}
\section{Extended model}\label{sec:change}
In this section we perform the new analysis on data collected from our
extended model where we include the behavior parameters in the culture vector. The results are shown in figure \ref{fig:cAll}.
Again, we limited the analysis to networks where the size of the LSCC
was at least 80\% of the original network.
This was $15700/21600$ samples or 73\%.
Compared with the same analysis of the
original model in figure \ref{fig:base-reduced},
the extended model here shows greater concentration around high
cultural diversity across all $\sd$ values.
This could be due to the fact that low cultural diversity networks
were removed due to low LSCC size.
However further investigation is warranted.

We also note that $SPL_s$ in the extended model in some trials is much higher than
in the original model. In fact, $SPL_s$ seems to increase as $\sd$ increases as seen in figure 6(a). Which is opposite of the original model's goals.

\subsection{Regression and Comparison with Original Model}

\begin{gather}
    <CD> ~ \thicksim ~ 2.27 + 1.75\sd + 1.32\ss - 0.18\sw -
                2.97\ds - 0.20\dw - 0.27\ssw \label{ca1}\\
    <SPL_s> ~ \thicksim ~ 2.34 + 1.05\sd + 0.13\ss + 0.11 \sw +
                0.28\ds - 0.004\dw + 0.14\ssw \label{ca2}
\end{gather}

Again, the equation for $<CD>$ remains similar to previous equations (\ref{one}),
and (\ref{red1}).
However, equation (\ref{ca2}) shows a similarity to the new analysis of the
original model in equation (\ref{red2}).
The terms show a large dependence on $\sd$, with a coefficient of $1.05$ for the
linear term as compared with $0.27$ in equation (\ref{red2}).
Meaning that increased diversity of cultural tolerance increases network
fragmentation.
Here we can conclude that our extended model shows greater fragmentation
of the network.
In addition, $\sd$ or diversity of cultural tolerance, correlates with larger
average shortest path length.

