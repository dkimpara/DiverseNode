\documentclass{article}

\usepackage{fullpage}
\usepackage{natbib}

% custom %
\usepackage{amsmath,amsthm,amssymb,amsfonts}
\usepackage{mathtools}
\usepackage{xcolor}
\usepackage[]{algorithm,algorithmic}
\usepackage{fancyhdr}
\usepackage{bbm}
\usepackage{enumitem}
% \usepackage{algorithm, algorithmic}
\usepackage{cleveref}
\usepackage{thm-restate}

\newcommand\tab[1][0.5cm]{\hspace*{#1}}
%
\theoremstyle{definition}
\newtheorem{defn}{Definition}
\newtheorem{theorem}{Theorem}
\newtheorem{lemma}{Lemma}
\newtheorem{proposition}{Proposition}

%
\newenvironment{claim}[1]{\par\noindent\underline{Claim:}\space#1}{}
\newenvironment{claimproof}[1]{\par\noindent\underline{Proof:}\space#1}

\DeclarePairedDelimiter{\ceil}{\lceil}{\rceil}
\DeclarePairedDelimiter{\floor}{\lfloor}{\rfloor}

\newenvironment{problem}[2][Problem]{\begin{trivlist}
		\item[\hskip \labelsep {\bfseries #1}\hskip \labelsep {\bfseries #2.}]}{\end{trivlist}}
% end custom %

\begin{document}

\title{Coexistence of Cultural Diversity}
 \author{Dhamma Kimpara}


\renewcommand{\thefootnote}{\arabic{footnote}}

\maketitle

%\begin{abstract}
%In this work, we investigate whether privacy and fairness can be simultaneously achieved by a single classifier in several different models. Some of the earliest work on fairness in algorithm design defined fairness as a guarantee of similar outputs for ``similar'' input data, a notion with tight technical connections to differential privacy. We study whether tensions exist between differential privacy and statistical notions of fairness, namely Equality of False Positives and Equality of False Negatives (EFP/EFN). We show that even under full distributional access, there are cases where the constraint of differential privacy precludes exact EFP/EFN. We then turn to ask whether one can learn a differentially private classifier which approximately satisfies EFP/EFN, and show the existence of a PAC learner which is private and approximately fair with high probability. We conclude by giving an efficient algorithm for classification that maintains utility and satisfies both privacy and approximate fairness with high probability.
%\end{abstract}

\section{Introduction}
\subsection{Overview}\label{subsec:overview}
    While homophily in social networks has been widely studied and observed,
    mechanisms to
    attempt to explain the emergence of homophily are equally important.
    One particular challenge, but also point of interest, in modeling
    network formation in this context are the simultaneous dynamics of
    formation/destruction of social ties and changes of individual states.
    Out of this conjunction of model attributes arises a complex process that can
    be thought of as a feedback loop between individual states and network structure.
    More generally this can be seen as the of the interaction of
    the micro (individual) and macro (structure) levels of a process.
    In other words, network structure depends on individual states and individual
    states depend on network structure.
    We seek to improve on stylized models that study one process abstracted from the other.

    Adaptive networks are the general class of dynamical networks whose topologies
    and states coevolve~\citep{sayama2013modeling}.
    In the context of cultural dis(integration) processes, we draw upon the
    adaptive network model in~\citep{social-frag} which investigates a mechanism that models the emergence of
    both cultural diversity and high connectivity.
    Previous models of adaptive social network dynamics did not demonstrate how
    such an outcome could occur.
    Either they showed homogenization and well connectedness or cultural diversity and
    fragmented structure.
    Though the model in~\citep{social-frag} has not yet been confirmed by empirical
    data, a different (much simpler) dynamic adaptive network model in~\citep{brot2012feedback}
    has been supported by empirical social network data.

    In this project we replicated the theoretical results in~\citep{social-frag},
    investigated an extension, and a potential weakness.
    The rest of the project is organized as follows: section\ref{sec:methods} describes
    the adaptive network model.
    Section \ref{subsec:summary} is a brief summary of our results.
    Section \ref{sec:confirm} confirms the results in \citep{social-frag}.
    Section \ref{sec:discrep} investigates a weakness in the original results
    and re-analyzes the original model.
    Section \ref{sec:change} anaylzes an extension of the model.




\newpage

\bibliographystyle{plainnat}
\bibliography{bib.bib}


\end{document}
