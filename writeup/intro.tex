
\subsection{Overview}\label{subsec:overview}
    While homophily in social networks has been widely studied and observed,
    mechanisms to
    attempt to explain the emergence of homophily are equally important.
    One particular challenge, but also point of interest, in modeling
    network formation in this context are the simultaneous dynamics of
    formation/destruction of social ties and changes of individual states.
    Out of this conjunction of model attributes arises a complex process that can
    be thought of as a feedback loop between individual states and network structure.
    More generally this can be seen as the of the interaction of
    the micro (individual) and macro (structure) levels of a process.
    In other words, network structure depends on individual states and individual
    states depend on network structure.
    We seek to improve on stylized models that study one process abstracted from the other.

    Adaptive networks are the general class of dynamical networks whose topologies
    and states coevolve~\cite{sayama2013modeling}.
    In the context of cultural dis(integration) processes, we draw upon the
    adaptive network model in~\cite{social-frag} which investigates a mechanism that models the emergence of
    both cultural diversity and high connectivity.
    Previous models of adaptive social network dynamics did not demonstrate how
    such an outcome could occur.
    Either they showed homogenization and well connectedness or diversity and
    fragmented structure.
    Though the model in~\cite{social-frag} has not yet been confirmed by empirical
    data, a different (much simpler) dynamic adaptive network model in~\cite{brot2012feedback}
    has been supported by empirical social network data.

\subsection{Methods}\label{subsec:methods}
    Following~\cite{social-frag}, we will simulate the dynamics of a social network
    with an initial configuration of two culturally distant groups.
    Each group
    consists of 50 individuals (nodes).
    Directed edges (representing information
    flow) are initiated at random with an edge density of 0.2 and 0.02, within group
    and between groups, respectively (directed stochastic block model).
    Edge weights are initiated by drawing from $\textrm{U}[0,1]$.
    Each individual's state or culture is represented by a vector in $\mathbb{R}^{10}$
    with cultural distance measured by the euclidean distance.
    Each individual will
    have a cultural update function and will update edges accordingly, with another
    function.
    There will be parameters, one each, for cultural tolerance, rate of cultural
    state change, and rate of edge weight change.
    For various settings of the parameters, we will run the model for a fixed number of
    iterations and plot the resulting average cultural distance between the two groups
    and average shortest path length in the network.

\subsection{Goals}\label{subsec:goals}
    We seek to confirm the theoretical results in~\cite{social-frag} and also
    investigate extensions and potential weaknesses.
    We will also investigate if this model generates networks that correspond
    to real social networks, in terms of small world property and clustering
    coefficient.
    For weaknesses, we would investigate the modeling assumption where an individual's
    culture is represented by ``culture" vector along with
    the seemingly arbitrary choice of cultural distance measured by the
    euclidean distance.
    To extend, we propose to use this model to observe the effect of cultural integration
    efforts such as team-building sessions in the business merger context.
    We will use the results from the confirmation section as a baseline.
    Another possible extension is to see what happens in the model if there are more
    than 2 initially culturally distinct groups.