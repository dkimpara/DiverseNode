\section{Methods}\label{sec:methods}

\subsection{Model}\label{subsec:model}
    Following~\citep{social-frag}, we will simulate the dynamics of a social network
    with an initial configuration of two culturally distant groups.
    The goal is to investigate what population attributes give rise to particular
    final configurations of the network in terms of cultural diversity and network structure.
    Particularly, we are interested in the case where there is both high
    cultural diversity and non-fragmented (highly connected) structure.

    In this model, each group consists of 50 individuals (nodes).
    Directed edges (representing information
    flow) are initiated at random with an edge density of 0.2 and 0.02, within group
    and between groups, respectively (directed stochastic block model).
    Edge weights are initiated by drawing from $\textrm{U}[0,1]$.

    To represent culture, each individual's state or culture is represented by a
    vector in $\mathbb{R}^{10}$ with cultural distance measured by the euclidean distance.
    This choice is based on empirical studies on measuring organizational culture.
    Each culture will have its own ``center" cultural vector, $C_i$.
    The two centers are initially
    separated by $3.0$ (arbitrary units) in the cultural space.
    Then for each individual in each group $i$, their cultural vector is drawn from
    $C_i + [N(0, 0.1), \dots , N(0, 0.1)]^T$.

    Now in each iteration of the simulation, each individual will interact with
    one other individual and update their culture and corresponding edge.
    Our interpretation was that the sequence of nodes performing actions would
    be randomized in each iteration.
    Each node, $v$, when it is their turn, performs as follows: first, with
    probability 0.99, $v$ selects $u$ from its in-neighbors with probability
    proportional to the edge weight.
    Or, with the remaining probability (0.01), $v$ selects any $u$ from the
    connected component in which $v$ belongs.
    Now, $v$ decides whether to accept or reject $u$'s culture.
    The probability of acceptance, $P_{A}$ is given by

    \[P_A = \frac{1}{2} ^ {\frac{|c_v - c_u|}{d}} .\]

    Where $c_v$ and $c_u$ are $v$ and $u$'s cultural vectors, and $d$ is the
    cultural tolerance, or the cultural distance at which $P_A = 0.5$.
    If $u$'s culture is accepted, $c_v$ is updated as

    \[c_v = (1-r_s)c_v + r_s c_u.\]

    Where $r_s$ is the rate of cultural change. The edge weight from $u$ to
    $v$, $w_{uv}$ is updated as

    \[w_{uv} = \text{logistic}(\text{logit}(w_{uv}) + r_w).\]

    Where $r_w$ is the rate of edge weight change. 
    However, if the $u$'s culture is rejected, $v$ does not update their culture
    but the edge weight is updated in the opposite direction as

    \[w_{uv} = \text{logistic}(\text{logit}(w_{uv}) - r_w)\]

    When an edge weight falls under 0.01, the edge is removed from the network.
    
    In all experiments, we use $d=0.5, r_s=0.5,$ and $r_w = 0.5$ as the mean
    values.
    In our extended model, we include $d$, $r_s$, and $r_w$ in
    culture vector $c_v$ for all individuals, hence $c_v \in \mathbb{R}^{13}$.

\subsection{Experiments}\label{subsec:experiments}
    For each experiment, we drew the behavior parameters of each individual,
    $d, r_s$ and $r_w$, from $N(0.5, \sigma_{-})$ where
    $\sigma_{-}$ corresponds to the parameter being sampled ($d, r_s$, or $r_w$).
    The standard deviations $\sd, \ss$, and $\sw$, were each varied
    between 0.0 and 0.5 at interval 0.1 (6 values for each standard deviation
    resulting in 216 unique settings).
    For each setting of the experimental standard deviations, we performed 100 independent
    trials.
    Each trial simulated 500 iterations of the process described in
    section \ref{subsec:model} and analyzed for the following statistics:

    \begin{defn}{Cultural Distance ($CD$).}
        Let $c_v$ be the culture vector of individual $v$ and $A_1, A_2$
        be the disjoint sets of individuals in each group in directed network $G$ then

        \[ <CD> := \frac{1}{|A_1||A_2|} \sum_{v \in A_1} \sum_{u \in A_2} |u - v|.\]

        Where $|\cdot|$ is the euclidean norm.
    \end{defn}

    \begin{defn}{Average Shortest Path Length ($SPL$).}

        \[ <SPL> := \frac{1}{n(n-1)} \sum_{v \neq u} d(u,v).\]

        Where $n$ is the number of nodes or individuals in $G$ and $d(u,v)$ is
        the length of the shortest path from $u$ to $v$, where all edges have
        weight/distance 1.
        By convention \citep{hagberg2008exploring}, $d(u,v)=0$ if $v$ cannot be reached
        from $u$.

    \end{defn}


    Here we note an ambiguity in the model: when $d, r_s$ and $r_w$ are drawn,
    they can be below zero or greater than one. 
    For certain parameters, such as $d$, one or both of these outcomes
    makes the parameter nonsensical.
    We chose to clip the values where appropriate in order for the model to make sense since
    \citep{social-frag} does not mention how to handle these cases.
    In experiments with high standard deviations $\sigma_{-} \thicksim 0.5$,
    this tends to cluster parameters at either $0.01$ or $0.99$.



\subsection{Summary of Results}\label{subsec:summary}
    First, we found that we were able to confirm the results when analyzing the data
    as described in \ref{subsec:experiments}.
    However, while doing this, an ambiguity in how to handle
    disconnected graphs presented a potential weakness.
    We investigated this ambiguity and found that when the analysis of the
    shortest path length was confined to only either the largest weakly
    connected component or the largest strongly connected component, the results
    in terms of SPL diverged from the original results.
    We then re-analyzed the original experiments and performed this same analysis
    on our extension where we include every individual's behavior parameters
    ($d$, $r_s$, $r_w$) in their culture vector. Note that we have omitted ANOVA 
    tables in order to not clutter the project.